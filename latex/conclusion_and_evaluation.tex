\chapter{Conclusion and Evaluation}

So far in this thesis many toppics have been covered. Chapter \ref{theory} covered theory related to snow plowing and routing problems. Then in chapter \ref{architecture_and_implementation} the implementation of the MA was discussed. In chapters \ref{cha:evolutionary_algorithm_configuration} and \ref{cha:experiments_and_results} the performance and outputs of the MA were covered.

This chapter will be focused on rounding off the thesis. It will start off by evaluating what questions have been answered and to what extent. Then the work with the thesis itself is going to be discussed. Afther that there will be a section on the contributions to the field of the work. And finally the future work that can be done to expand upon what has been covered in this thesis will be outlined.

\section{Evaluation}
% Evaluation of results

% RQ1 was answered by...
% RQ2 was answered by ... in section ...

% All in all it can be concluded that given a good model, snow plowing routing in Trondheim (or the entirety of norway for that matter) is possible and feasible.

When evaluating the work in this thesis the research questions should be considered. The first research question was stated in chapter \ref{cha:introduction} as \emph{\enquote{Obtain and present the related work which gives the context for this thesis}}. It was subsequently treated in chapter \ref{theory}, which covered both the background for snow plowing and the theory surrounding routing problems. Section \ref{sec:routing_algorithms} discussed why a MA was chosen as the approach to the routing problem, and later in section \ref{sec:evolutionary_algorithms} MAs were explained more in details. Then in section \ref{sub:fitness_for_snow_plowing} the theory on how to adapt a MA better to the task of snow plowing was dealt with. Finally in sections \ref{sub:all_pairs_shortest_path} and \ref{sub:the_split_algorithm} the reasoning behind the adaption of a couple of known algorithms was laid out. Thus chapter \ref{theory} has presented related work, tied it to the setting of the thesis, and thereby answered RQ1.

% \rowcolors{1}{gray!15}{white}
% \begin{table}[H]
% \centering
% \begin{tabular}{cp{0.8\textwidth}}
% RQ1  &  Obtain and present the related work which gives the context for this thesis. \\
% RQ2  &  Configure the memetic algorithm so that it performs well.\\
% RQ3  &  Determine whether the memetic algorithm independently can find a route in Trondheim optimized for length weighted by speed with a random starting point. \\
% \end{tabular}
% \caption{Research questions}
% \label{tab:research_questions}
% \end{table}

% \begin{description}
% 	\item [Goal] Make a system that can generate optimal routes for snow plowing in Trondheim.
% \end{description}

\section{Discussion}
Discussion of results

Model could be better
Implementation could be better (more paralell and generally optimal)



\section{Contributions}
How do our findings contribute to the field. Of what signifficance are our findings?

\section{Future Work}
What we didn't do but know that could be done

\subsection{Improve Performance}
Much of the code could be parallized, perhaps altered so that it can run on gpu-s

Reduce constant factors (of running time (ex some loops could be n\^2 instead of 2\*n\^2)).

\subsection{Better Heuristic}
Several vehicle types and ability to divide into zones

Take todays weather into consideration (ie. make a real-time system, and differentiate between emergency servicing and maintenace work).

Give output in terms of not only what we currently do, but also total distance, amount of fuel spent, time spent, vehicle durability (how much is their lifespan reduced by by doing the suggested route), or environmental impact. This will allow the operator to choose what to optimize, and it could even be a weighting among all the different evaluation types. Input could then for an instance be fuel > time > distance > environmental impact, or 90\% environment, 80\% fuel, 85\%time, etc.

\cleardoublepage
