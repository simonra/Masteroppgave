\chapter{Mathodology and Implementation}

\subsection{Architectural Overview}
Main components. NVDB and EA, written in python/java, overall workflow.

\subsubsection{System Workflow}
This is what happens when the system is run:

\begin{itemize}
	\item The system fetches the map of Sør-Trøndelag County from NVDB.
	\item The user has to specify a sub-area to be processed.
	\item The given area is converted to a standard NEARP format (http://www.sintef.no/projectweb/top/nearp/documentation/) and written to a file.
	\item The user has to specify what parameters to use for the evolutionary algorithm.
	\item The tour calculating model produces an internal graph representation from the file with the map area, and an all pairs shortest path matrix is made.
	\item The evolutionary algorithm processes the graph and makes a suggested solution based on the users parameters.
	\item The solution is drawn on a map for readability.
\end{itemize}

\subsection{NVDB}

\subsection{Internal Data Model}
Reads graph of standard format -> Finds all pairs shortest paths using floyd warshall (mention the neat trick of including destination cost)

\subsection{Evolutionary Algorithm}
input: tweakable param files -> pseudocode presented as text -> mention about the memetic part

\subsection{Heurisitc}
What we implemented 
-> arguing for why these are good and importat things 
-> mentioning that this doesn't solve the multi vehicle split that is oh so standard in the literature (because not relevant for our 1 car use case) 
	-> It could possibly be used to solve some other of trondheims probles though 
-> mentioning what we don't take into consideration, and why one might want to look into that (ex impact of weather, slope of roads, curvature, etc)

\cleardoublepage
