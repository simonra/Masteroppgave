\chapter{Methodology and Implementation}

In the previous chapter the foundation for implementing a system that does route optimization for snow plowing has been outlined. It has been shown why the NEARP has been chose as the underlying model. Further it was argued why MAs are a good approach to the problem. Then issues relating to the combination of the snow plowing problem and MAs solving the NEARP were discussed.

Now with the background for the implementation sorted out, the architecture should be considered.






We have what the system should do (snow, NEARP) and how it should be done (MA, special fitness).
Here is the architecture of the system we based upon this.
	Overall process
	External resources and output
MA details
	detailed loop
	what we focues on in the fitness implementation
Expected asympthotical running time
	parse graph |g| + flw -> |g|\^3 + MA -> generations * (number of parents * selection overhead + number of children * (generating overhead (|genome|) + fitness evaluation of each (if not split |genome/graph| else if split |genome\^2|) ) + overhead of population transitions (sorting the population (n*log n ?), selecting the next population (O(1) in our implementations), updating best knowns (O(1) also)))
General maintainability of the implementation
Expected future reliance of the external resources used (NRDB, sos-files)
On to how to set up experiments for this.




% Now it is time to discuss the implementation.
% The next thing that should be discussed is the implementation.
% Once all of that is out of the way, next the implementation should be discussed.
% At this point, the implementation should be discussed.
% This brings us to how the system has been/was implemented.
% The system that sprang out of this ...
% Now that this has been discussed, 
% Now that that has been covered, it is time to look at the implementation.
% Now when we have covered that, ...
% Now that the background has been covered, one should look at the implementation
% What all of this has lead to *is the solid foundation of what comes next, the implementation*
% Now that all of that is behind us, it is time to look at the actual work behind this thesis
% All of this/that comes together in the implementation
% The implementation uses all of those things/that stuff
% The outcome of all that hard work was the implementation
% And here it is. The implementation.
% Now on to the implementation.
% The architecture that resulted
% The resulting architecture is what will be discussed next.
% All of that hard work, what for? Well here is the architecture...
% What is needed now is an explanation of the architecture of the implementation that was made 
% With the motivation and for and reasoning behind the implementation sorted out, 

% \subsection{Architectural Overview}
% Main components. NVDB and EA, written in python/java, overall workflow.

% \subsubsection{System Workflow}
% This is what happens when the system is run:

% \begin{itemize}
% 	\item The system fetches the map of Sør-Trøndelag County from NVDB.
% 	\item The user has to specify a sub-area to be processed.
% 	\item The given area is converted to a standard NEARP format (\url{http://www.sintef.no/projectweb/top/nearp/documentation/}) and written to a file.
% 	\item The user has to specify what parameters to use for the evolutionary algorithm.
% 	\item The tour calculating model produces an internal graph representation from the file with the map area, and an all pairs shortest path matrix is made.
% 	\item The evolutionary algorithm processes the graph and makes a suggested solution based on the users parameters.
% 	\item The solution is drawn on a map for readability.
% \end{itemize}

% \subsection{NVDB}

% \subsection{Internal Data Model}
% Reads graph of standard format -> Finds all pairs shortest paths using floyd warshall (mention the neat trick of including destination cost)

% \subsection{Evolutionary Algorithm}
% input: tweakable param files -> pseudocode presented as text -> mention about the memetic part

% \subsection{Heurisitc}
% What we implemented 
% -> arguing for why these are good and importat things 
% -> mentioning that this doesn't solve the multi vehicle split that is oh so standard in the literature (because not relevant for our 1 car use case) 
% 	-> It could possibly be used to solve some other of trondheims probles though 
% -> mentioning what we don't take into consideration, and why one might want to look into that (ex impact of weather, slope of roads, curvature, etc)

\cleardoublepage
