\chapter{Experiments and Results}
\label{cha:experiments_and_results}

In this chapter obtaining results from the MA will be considered. First there will be a section that looks into the results one gets when the MA is applied to the BHW1 benchmark \citep{BHWdocumentationSINTEF}. Because the smaller BHW instances are solved to optimality, this will give an idea of the MAs performance relative to known optimal solutions. Then there will be a section about obtaining results when map data from Trondheim is used as input to the MA.

\section{The BHW Benchmark}
\label{sec:the_bhw_benchmark}

Here the MAs performance on the BHW1 test case is going to be presented. Because the BHW1 benchmark has been solved optimally, and the solution is available, it can be used to verify the output of the MA. When comparing the fitness of the result from the MA with the fitness of the know solution one should be able to gauge the quality of the MAs output. And when observing the optimal route with the one made by the MA it should become apparent whether the MA is on the right track or not.

The section will start out by outlining the experimental plan for the tests. After the experimental plan has been given, the experimental setup will be described. And finally the results of the tests will be presented.

\subsection{Experimental Plan}

To be able to compare the routes generated by the MA and the known solution for the BHW1 case and their fitnesses, clearly a result form running the MA on the BHW1 benchmark has to be obtained. The generated route and the known solutions' route should be displayed side by side, and the fitness the MA finds should be compared to the fitness of the solution.

\subsection{Experimental Setup}

To be able to compare the fitness and the rsulting routes produced by the MA with the known solution, the MA should be set to process the BHW1 case. The result should be compared to the one found on \citet{BHW1Solution}.

To get an idea of what fitnesses the MA genereally finds when processing the BHW1 case, it will be run 30 times for 50000 generations with the parameters shown in table \ref{tab:BHW1_params_table}. The fitness evaluation will be performed by evaluating the sum of the lengths of the splitted trips generated from each genome, due to that being how the fitness of the optimal solution is calculated. After having preformed the series of runs, the result from the best run is going to be picked and compared to the known solution.

{
\rowcolors{2}{gray!15}{white}
\begin{table}[tbph]
\centering
\begin{tabular}{ll}
\hline
\textbf{MA Parameter} & \textbf{Setting to be Used}     \\ \hline
Parent selection      & Fitness proportionate selection \\
Adult selection       & Overproduction                  \\
Mutation type         & Memetic improvement             \\
Population size       & 90 individuals                  \\ \hline
\end{tabular}
\caption{The parameters used for testing the MA on the BHW1 benchmark.}
\label{tab:BHW1_params_table}
\end{table}
}

\subsection{Results}

After running the MA on the BHW1 benchmark with the parameters described above, the average of the best, average, and standard deviation of fitness of the populations each generation were plotted. The charts can be seen in appendix \ref{cha:bhw1_benchmark_results}. From the cart in figure \ref{fig:bhw1ab}, it can be seen that the best solutions the MA finds have a fitness of about 370.

The best result obtained was the run labeled as 2015-06-24T01-34-22Z, which can be found in the supplementary digital materials. It had a fitness of 360, which compared to the optimal fitness that is 337 is completely fine. The trips found by the MA and and the optimal trips from \citet{BHW1Solution} can be seen in table \ref{tab:BHW1_solutions_compared}. As can be seen in the table, the trips found by the MA show resemblance to those of the optimal solution. For an instance the first part of the fourth trip found by the MA is very similar to the first part of the third trip of the optimal solution.

{
\rowcolors{2}{gray!15}{white}
\begin{table}[tbph]
\centering
\resizebox{\textwidth}{!}{
\begin{tabular}{lll}
\toprule
                & \textbf{Result from MA}                         & \textbf{Known Optimal Solution} \\ \midrule
\textbf{Trip 1} & 1-A5-N12-E8-N12-A9-6-E4-6-N12 1                 & 1-A1-N2-E3-9-A11-11-E5-5 1     \\
\textbf{Trip 2} & 1-A1-N2-E3-9-A11-N11-E9-8-A10-N10-E10-9-E3-N2 1 & 1-A2-4-E2-2-E1-N3 5-E6-12 1    \\
\textbf{Trip 3} & 1-A2-N4-E2-N2-E3-N2-E1-N2 1                     & 1-A4-10-E11-N11-E9-8 7-E8-12 1 \\
\textbf{Trip 4} & 1-A4-N10-E11-N11-E5-5-E4-6-N12 1                & 1 12-A9-6-E4-5-A7-3-A6-N4 1    \\
\textbf{Trip 5} & 1-A3-N7-E7-N7-A8-6-N12 1                        & 1-A3-7-E7-8-A10-N10-E10-9 1    \\
\textbf{Trip 6} & 1-A5-N12-E6-5-A7-N3-A6-N4 1                     & 1-A5-N12 N7-A8-6 1             \\ \midrule
\textbf{Fitness}    &    \textbf{360}    &    \textbf{337}    \\ \bottomrule

\end{tabular}
} %close resizebox
\caption{The best result obtained from the MA and the known optimal solution for the BHW1 benchmark.}
\label{tab:BHW1_solutions_compared}
\end{table}
}

\section{Map Data from Trondheim}

Now that it has been established that the MA works (as shown in chapter \ref{cha:evolutionary_algorithm_configuration}), and produces results of a reasonable quality when applied to benchmark problems (detailed in section \ref{sec:the_bhw_benchmark}), RQ3 should be addressed. RQ3 was stated in table \ref{tab:research_questions} as: \enquote{Determine whether the memetic algorithm independently can find a route in Trondheim optimized for length weighted by speed with a random starting point}.

To properly test whether the MA can routes in Trondheim, it should be applied to map data form Trondheim. This section will start out by exploring what data should be gathered to properly deal with RQ3. Then the experiments used to obtain the data will be detailed. Finally after that the results obtained from the experiments will be outlined.

\subsection{Experimental Plan}

To get results on the performance of the MA when trying to create snow plowing routes for Trondheim, clearly it should be fed a graph that is generated from road map data of Trondheim. To check whether the MA is consistently capable of finding solutions it should be run a number of times, and the quality of the results in terms of their fitness should be plotted.

Once the MA has generated a set of results, these results should be compared to the routes currently being driven. To do this the fitnesses of the generated routes should be compared to the fitness the implementation calculates for the routes as they are driven. This way they will both be evaluated in terms of a common refference frame, the model based on the map data from Trondheim.

Then the routes generated by the MA should be drawn up as a map. It could be argued that they should be drawn \emph{on} a map, but due to copyright issues and the nature of the process of generating map tiles it is not feasible to do at this point in time. When the routes have been drawn up, they can visually be inspected for flaws and abnormalities. Additionally having the routes presented graphically is usefull if they are to be applied in a real life setting by a snow plow driver.

The drawn routes generated by the MA could also then be evaluated by the municipalitys drivers. This could bring to light aspects of the generated routes that might otherwise be overlooked, and their assessment can also be used as a way to evaluate the quality of the generated routes.

\subsection{Experimental Setup}
Run the MA a set number of times on the trondheim data, with the configuration found in the tuning chapter. When the output is given, strip it to the id-s without the extra nearp stuff, put it in a csv, and feed it into qgis. Draw a map, add hot water, and present results.
Explain how the existing routes are obtained and fed into the system.

\subsection{Results}
Show some nice graphs of how the fitness converges over time.
Show a created map, and refer to tiles with finer details in the appendix.
Show how the fitness of the pre-existing routes relates to the fitness of the calculated ones.
Discuss how the system does som smart things that humans would perhaps not (cover the city in a set of veird overlapping sections), how some interesting behavior like driving into a road, servicing it, and then driving back out the same end as it was started in and continue that route arises. Also highlight some missed one way roads and gaps in the map.


\cleardoublepage