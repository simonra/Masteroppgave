\chapter{Introduction}

\section{Background and Motivation}

The background for this thesis was that the municipality of Trondheim (in Sør-Trønderlag county, Norway) wanted to look into improving the performance of their winter road maintenance work. We were approached with the challenge of optimizing the routes they use for their snow plowing operation. After some initial meetings with our supervisors, the municipality, and the Norwegian Public Roads Administration (NPRA), we had determined the scope of our pre-project which took place in the autumn of 2014.

Our initial research showed that routing is a hard problem and that even with heuristic an exact solution is not guaranteed. Furthermore our resources were limited to one, maybe two semetsters. This contributed to that we soon realized that any solution of sufficient quality we might produce would not be a complete answer to the municipalitys challenge.

An ideal solution would be one that looked at the entrie municipality, divided it into routes for each available vehicle (taking into account both the municipalitys vehicles and those of contractors), while making sure that each route is optimal for the chosen vehicle and that all the required area gets covered. At the same time it should take into consideration the features of each individual piece of road, such as slope, width, number of pedestriant crossings, and what parameters the user wants to optimize for. The durability loss of the vehicles, total distance moved for each vehicles, total time the drivers spend, the environmental impact of the operation, how trafic flow is affected, are all important factors one might want to optimize, and how one weights them may greatly affect the outcome.

The outcome on the other hand, is specified by the NPRA. In their manuals they have detailed what conditions should be like, how much time can pass before roads of certains types have to be restored to normal operational standards, and who are responsible for what kinds of roads. Totally they specify X kinds of roads. In decreasing order of importance they are:

	Hver vegtype med korte beskrivelser

The NPRA is responsible for maintaining the TODO. The responsibility for X, Y, and Z types of roads befall the individual municipalities. Lastly the responsibility for private roads befall the (private) owners. This means that our routing should not take into account servicing T types of roads, but that the municipality is interested in working on pavements and bicycle-paths as well as the road network.

SVV har data (NVDB)

Innen AI er det CPP, (NE)ARP (Ref forprosjekt)

\section{Goals and Research Questions}

\section{Reseach Method}

\section{Thesis Structure}



\cleardoublepage
