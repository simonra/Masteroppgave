\chapter{Introduction}



\section{Background and Motivation}



\section{Goals and Research Questions}




\rowcolors{1}{gray!15}{white}
\begin{table}[tbph]
\centering
\begin{tabular}{cp{0.8\textwidth}}
RQ1  &  Obtain and present the related work which gives the context for this thesis. \\
RQ2  &  Configure the memetic algorithm so that it performs well.\\
RQ3  &  Determine whether the memetic algorithm independently can find a route in Trondheim optimized for length weighted by speed with a random starting point. \\
\end{tabular}
\caption{Research questions}
\label{tab:research_questions}
\end{table}

\section{Reseach Method}

To address RQ1 large parts of chapter \ref{theory} will be dedicated to presenting the relevant related work which was found in the pre-project of this thesis. RQ2 will be treated in chapter \ref{cha:evolutionary_algorithm_configuration}, where the parameters are tuned in a set of quantitative experiments. Finally RQ3 is considered in chapter \ref{cha:experiments_and_results}. Here there will be a set of quantitative experiments to determine the quality of the routes produced by the memetic algoritm compared to the routes currently being driven. The comparison will be in terms of how the algorithms fitness function perceives them. After the results have been generated and evaluated by the system, the best that have been found will be shown to domain experts at the municipality. They will then give their assessment on the quality of the routes.

\section{Thesis Structure}

The first chapter after the introduction, named Theory, will start out by giving the neccessary background on snow plowing. After that the related work will be presented. In the following chapter, Architecture and Implementation, details of design decisions and important details of the implementation will be given. The next chapter, Evolutionary Algorithm Configuration, will outline the process of how the memetic algorithm was tuned and what parameters were found to give the best performance. In the Experiments and Results chapter which comes next, the performance of the memetic algorithm will be tested, and the experimental setup for how it processed Trondheim will be shown. Whereupon the resulting data will be presented and discussed. Finally, in the Evaluation and Conclusion chapter the entire process will be evaluated, and conclusions drawn from the results presented. At the very end the future work will outline how the work in this thesis can be extended.

\cleardoublepage
