\chapter{Introduction}

This thesis will be about route optimization for snow plowing. It will handle it by looking into what snow plowing is, how it can be modelled and processed, and discussing how the chosen approach seems to work out. In this chapter the background and motivation for doing route optimization for snow plowing will be explored, the goals and research questions for this thesis are going to be set out, and the structure of the rest of the text will be outlined.

\section{Background and Motivation}

In places that get a lot of snow during the winter, snow removal is an important logisitcal challenge. If not done it severely affects the usability of the road infrastructure. And the removal process itself can also adversely impact the flow of traffic. It is therefore in the municipalitys interest to do the removal as efficiently as possible. Both in terms of how quickly the entire road network can be cleared, as well as how to do it so as to not be an obstacle in the traffic. On top of that the municipality has financial interests in minimizing the working hours spent and durability suffered by the equipment. Also it could be interested in reducing the environmental impact of the operations in terms of emissions due to the driving, as well as particulates released by wear and tear on the road due to the work (not to mention the environmental impact of replacing worn out road).

To handle this, the municipality wanted to look into whether the routes they drive could be improved using modern computational methods. This paper tries to address that concern by applying artificial intelligence methods to the map data from the .

\section{Goal and Research Questions}

The main goal for this thesis can be stated as follows:

\begin{description}
	\item [Goal] Make a system that can generate optimal routes for snow plowing in Trondheim.
\end{description}

In the context of this thesis, system refers to a set of scripts and programs. Most importantly it encompasses the module that is responsible for dealing with road-map data, and a memetic algorithm that will process the map data to yield routes. Optmial refers to that the route is as short as possible in terms of length weighted by speed.

The first research question that should be addressed in order to to reach this goal is obtaining and presenting the relevant related work that gives the context for this thesis. Answering this might show whether optimal routes are possible to find at all, and why a memetic algorithm was chosen as the apporach.

The second research question that should be answered given that a memetic algorithm has been chosen, is how to configure it so that it performs well. Memetic algorithms, being closely related to genetic algortihms, will perform horribly if intialized with the wrong parameters, and finding good ones is non-trivial. Therefore not only the parameters themselves, but also the process of obtaining them should be shown.

Finally in addessing the goal there needs to be a research question that asks to determine whether the memetic algorithm independently can find a route in Trondheim optimized for length weighted by speed with a random starting point. Answering this should reveal whether the goal has been completely reached, what (if any?) areas are left to explore in the future work, and whether the memetic algorithm performs satisfactory for the application.

\rowcolors{1}{gray!15}{white}
\begin{table}[H]
\centering
\begin{tabular}{cp{0.8\textwidth}}
RQ1  &  Obtain and present the related work which gives the context for this thesis. \\
RQ2  &  Configure the memetic algorithm so that it performs well.\\
RQ3  &  Determine whether the memetic algorithm independently can find a route in Trondheim optimized for length weighted by speed with a random starting point. \\
\end{tabular}
\caption{Research questions}
\label{tab:research_questions}
\end{table}

\section{Reseach Method}

To address RQ1 large parts of chapter \ref{theory} will be dedicated to presenting the relevant related work which was found in the pre-project of this thesis. RQ2 will be treated in chapter \ref{cha:evolutionary_algorithm_configuration}, where the parameters are tuned in a set of quantitative experiments. Finally RQ3 is considered in chapter \ref{cha:experiments_and_results}. Here there will be a set of quantitative experiments to determine the quality of the routes produced by the memetic algorithm compared to the routes currently being driven. The comparison will be in terms of how the algorithms fitness function perceives them. After the results have been generated and evaluated by the system, the best that have been found will be shown to domain experts at the municipality. They will then give their assessment on the quality of the routes.

\section{Thesis Structure}

The first chapter after the introduction, named Theory, will start out by giving the neccessary background on snow plowing. After that the related work will be presented. In the following chapter, Architecture and Implementation, details of design decisions and important details of the implementation will be given. The next chapter, Evolutionary Algorithm Configuration, will outline the process of how the memetic algorithm was tuned and what parameters were found to give the best performance. In the Experiments and Results chapter which comes next, the performance of the memetic algorithm will be tested, and the experimental setup for how it processed Trondheim will be shown. Whereupon the resulting data will be presented and discussed. Finally, in the Evaluation and Conclusion chapter the entire process will be evaluated, and conclusions drawn from the results presented. At the very end the future work will outline how the work in this thesis can be extended.

\cleardoublepage
