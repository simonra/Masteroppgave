\chapter{Theory}

\section{Background Theory}
In the pre-project leading up to this work we performed a structured literature review (TODO: SOURCE). Our focus was finding whether snow plowing was a known and solved problem, and in case it was, how should one go about trying to solve it. In the following sections we will outline what we found about the classification of the problem, and known approaches.

\subsection{The Node, Edge, and Arc Rouring Problem(s)}
In our structured literature review we found that the current take on problems like snow plowing has long roots. The simplest possible interpretation is that it is about traversing a graph G=\{N,E\} in a certain way. Which would make it similar to the “Seven bridges of Königsberg Problem”, that is about finding a cycle that traverses the graph and visits all the edges exactly once, and was solved by the Swiss mathematician Leonhard Euler in 1735.

But it was not untill more than 200 years later (in 1962) that a graph traversal problem that is more relevant to snow plowing was proposed. The chinese mathematician, and former postman, Mei-Ko Kwan described the problem of finding the shortest trip that visits all the edges in a graph at least once. It is the converse of the Vehicle Routing Problem (VRP), which is about visiting all the nodes of the graph at the lowest possible cost. The problem became known as the Chinese Postman Problem.

The CPP, if it contains only undirected edges or directed edges (arcs) (G=\{N,E\} or G=\{N,A\}), can be solved in polynomial time. But if it is a mixed graph containing both (G=\{N,E,A\}) it is NP-hard.

Over the following decades several variations to the probem were made to handle speciffic challenges. We will now briefly define the most known formulations and how they are tied to snow plowing.

The Rural Postman Problem (RPP) arose when one wanted to optimize solutions for when the postman does not have to visit all the edges, but only a subset of them. Which is a situation that is likely to arise in snow plowing, where one for an instance has to service all the roads in a city with a certain amount of trafic, but can pass through the less used roads to move between the ones that have to be plowed.

The Mixed Chinese Postman Problem (MCPP) is the apporach to the CPP where one looks at the problem in a mixed graph (with both arcs and edges). It introduces a relevant constraint when considering practical applications, such as snow plowing, where the underlying graph represents a road network. Not only is it a very natural representation of a road network containing both bi-directional and one way roads, it can also be used to handle cases such as intersections with forbidden turns. 

The Hierarchical Chinese Postman Problem (HCPP) came about when one wanted to investigate how solutions would change if one required some edges to be serviced before others. It is a situation that often arises in practice. For an instance in snow plowing one might want to service a more used road before a less used road.

The Windy Postman Problem (WPP) is the attempt at accounting for that an edge can have a different cost each time you pass through it. This is a relevant consideration when snow plowing, because simply passing through a road and plowing it takes a different amount of resources and time, and even passing through a road that has already been plowed and one that has not can have separate costs.

The k-postman Chinese Postman Problem (k-CPP) is the attempt at modeling a scenario where there is a depot node, and k postmen, or one postman that can do k trips of a certain cost. Taking it a step further, the Min-Max k-CPP (MM k-CPP) looks at minimizing the the trip with the maximal cost of the k trips. This is relevant for the snow plowing case where there is a large road network and there is no single vehicle that is capable of service all of it in one go. If implemented correctly it can even be argued that the splitted trips from the k-CPP can be used for sectoring the road network between different contractors.


% History:
% Euler -> Mei ko kwan -> CPP -> RPP -> MCPP -> HCPP -> WPP -> (mm)k-CPP-> ARP -> CARP -> NEARP(/ECARP and variations)

\subsection{Evolutionary Algorithms}
Why we chose this approach
	(Many approaches to the problem have been tried, examples, however in our previous work we found that MA's were good)
\\\\
Description of:
Turing -> modern interpretations
\\\\
Memeticism

\cleardoublepage