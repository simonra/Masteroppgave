\clearpage
\pagenumbering{roman}                 
\setcounter{page}{1}

\pagestyle{fancy}
\fancyhf{}
\renewcommand{\chaptermark}[1]{\markboth{\chaptername\ \thechapter.\ #1}{}}
\renewcommand{\sectionmark}[1]{\markright{\thesection\ #1}}
\renewcommand{\headrulewidth}{0.1ex}
\renewcommand{\footrulewidth}{0.1ex}
\fancyfoot[LE,RO]{\thepage}
\fancypagestyle{plain}{\fancyhf{}\fancyfoot[LE,RO]{\thepage}\renewcommand{\headrulewidth}{0ex}}

\section*{\Huge Abstract}
\addcontentsline{toc}{chapter}{Abstract}    
$\\[0.5cm]$

In this thesis route optimization for snow plowing in Trondheim municipality (Norway) is explored. It is modeled as a node, edge, and arc routing problem, which has been shown to be NP-hard. To generate the routes a memetic algorithm is used, which is a type of genetic algorithm where instead of mutation a simple local search to improve each genome can be performed. The routes that are generated by this algorithm are optimized for being as short as possible weighted by speed limits, so that longer stretches that can be passed through faster are preferred over shorter ones with a lower speed limit.

When working on the routes, map data from the municipality and the Norwegian Public Roads Administration is used. To verify the routes that are generated, they are compared with how the routes that cover the same set of roads are currently being driven by processing both with the algorithms fitness function.

The results show that the algorithm is capable of finding better routes than the ones that are currently driven in terms of the used fitness function, but is never capable of finding the optimal solution in simple constructed test cases. The fitness function however suffers from that it is rather simple, and does not take into account important factors such as the width of the roads and what that means for how many times a given road has to be traversed for it to be sufficiently clean.

What that means is that while the routes generated are not good for practical use (even though the drivers seem to like the perspective a third party suggestion gives on their work), they show that it is possible to draw up routes with better fitness than what is being driven by humans, and that processing routes for areas with the same amount of roads as the city center in Trondheim is feasible.



% Begin Norwegian abstract
\clearpage

\section*{\huge Sammendrag (Abstract in Norwegian)}
\addcontentsline{toc}{chapter}{Sammendrag (Abstract in Norwegian)}    
$\\[0.5cm]$

I denne masteroppgaven er ruteoptimalisering for snøbrøyting i Trondheim kommune (Norge) utforsket. Det er modellert som et node, kant, og rettet kant-ruting problem, som er kjent å være NP-hardt. For å generere rutene er en memetisk algoritme brukt, som er en type genetisk algoritme der man kan gjøre et enkelt lokalt søk for å forbedre hvert genom i stedet for mutasjon. Rutene som genereres med denne algoritmen er optimalisert for å være så korte som mulig vektet med fartsgrenser, slik at lengre strekninger som kan kjøres gjennom kjappere foretrekkes fremfor kortere med laver fartsgrense.

For å prosessere rutene er kartdata fra kommunen og Statens Vegvesen brukt. For å verifisere rutene som genereres, blir de sammenlignet med hvordan rutene som dekker det samme settet av veier blir kjørt nå av brøytebilførerene med algortimens fintess-funksjon.

Resultatene viser at algoritmen er i stand til å finne bedre ruter enn de som for tiden kjøres når de vurderes med fintess-funksjonen, men den er ikke i stand til å finne den optimale løsningen i enkle konstruerte testproblemer. Fintess-funksjonen lider imidlertid av at den er ganske simpel, og ikke tar hensyn til viktige variabler slik som bredden på veiene, og hva den har å si for hvor mange ganger en gitt vei må traverseres for at den skal bli tilstrekkelig fri for snø.

Hva dette betyr er at de generete rutene ikke er gode for praktisk bruk (selv om sjåførene synes å like perspektivet de gir på arbeidet deres), viser de at det er mulig å lage ruter med bedre fitness-verdi enn ruter kjørt av mennesker, og at å generere ruter for områder med samme mengde veier som Trondheim sentrum er gjennomførbart i praksis.

% End Norwegian abstract


\cleardoublepage
