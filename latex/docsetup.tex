\usepackage{setspace}
\usepackage{graphicx}
\usepackage{amssymb}
\usepackage{mathrsfs}
\usepackage{amsthm}
\usepackage{amsmath}
\usepackage{color}
\usepackage[Lenny]{fncychap}
\usepackage[pdftex,bookmarks=true]{hyperref}
\usepackage[pdftex]{hyperref}
\hypersetup{
    colorlinks,%
    citecolor=black,%
    filecolor=black,%
    linkcolor=black,%
    urlcolor=black
}
\usepackage[T1]{fontenc}
\usepackage[utf8]{inputenc}
\usepackage[font=small,labelfont=bf]{caption}
\usepackage{fancyhdr}
\usepackage{times}
%\usepackage[intoc]{nomencl}
%\renewcommand{\nomname}{List of Abbreviations}
%\makenomenclature
% Remember that texlive-publishers has to be installed for the following one to work!!
\usepackage{natbib}
\setcitestyle{square}
\usepackage{float}
%The two following packages requires that on has installed texlive-science
\usepackage{algorithm}
\usepackage{algpseudocode}
%Toggle this and the one above to have the endFor and endIf statements or not
% \usepackage[noend]{algpseudocode}
%Breaks url's in sensible ways across lines, and makes them clickable.
\usepackage{url}
% Used for getting more compact lists
\usepackage{paralist}
\usepackage{enumitem}

\usepackage{wrapfig}
\usepackage[table]{xcolor}
\usepackage{microtype}
\usepackage{pdflscape}
% \usepackage{tablefootnote}
\usepackage{todonotes}

%\floatstyle{boxed}
\restylefloat{figure}

% \usepackage{glossaries}
% \makeglossary
% \newglossarytype[abr]{abbr}{abt}{abl}
% \newglossarytype[alg]{acronyms}{acr}{acn}
\newcommand{\abbrname}{Abbreviations} 
\newcommand{\shortabbrname}{Abbreviations}
%\makeabbr
\newcommand{\HRule}{\rule{\linewidth}{0.5mm}}

\renewcommand*\contentsname{Table of Contents}

\pagestyle{fancy}
\fancyhf{}
\renewcommand{\chaptermark}[1]{\markboth{\chaptername\ \thechapter.\ #1}{}}
\renewcommand{\sectionmark}[1]{\markright{\thesection\ #1}}
\renewcommand{\headrulewidth}{0.1ex}
\renewcommand{\footrulewidth}{0.1ex}
\fancypagestyle{plain}{\fancyhf{}\fancyfoot[LE,RO]{\thepage}\renewcommand{\headrulewidth}{0ex}}

% Header style
% \chapterstyle{wilsondob}

% Allows usage of Do While with algorithmicx:
\algdef{SE}[DOWHILE]{DoWhile}{EndDoWhile}{\algorithmicdo}[1]{\algorithmicwhile\ #1}%http://tex.stackexchange.com/questions/115709/do-while-loop-in-pseudo-code